% Digital Logic Report Template
% Created: 2020-01-16 Sebastian Lopez 

%==========================================================
%=========== Document Setup  ==============================

% Formatting defined by class file
\documentclass[11pt]{article}

% ---- Document formatting ----
\usepackage[margin=1in]{geometry}	% Narrower margins
\usepackage{booktabs}				% Nice formatting of tables
\usepackage{graphicx}				% Ability to include graphics

%\setlength\parindent{0pt}	% Do not indent first line of paragraphs 
\usepackage[parfill]{parskip}		% Line space b/w paragraphs
%	parfill option prevents last line of pgrph from being fully justified

% Parskip package adds too much space around titles, fix with this
\RequirePackage{titlesec}
\titlespacing\section{0pt}{8pt plus 4pt minus 2pt}{3pt plus 2pt minus 2pt}
\titlespacing\subsection{0pt}{4pt plus 4pt minus 2pt}{-2pt plus 2pt minus 2pt}
\titlespacing\subsubsection{0pt}{2pt plus 4pt minus 2pt}{-6pt plus 2pt minus 2pt}

% ---- Hyperlinks ----
\usepackage[colorlinks=true,urlcolor=blue]{hyperref}	% For URL's. Automatically links internal references.

% ---- Code listings ----
\usepackage{listings} 					% Nice code layout and inclusion
\usepackage[usenames,dvipsnames]{xcolor}	% Colors (needs to be defined before using colors)

% Define custom colors for listings
\definecolor{listinggray}{gray}{0.98}		% Listings background color
\definecolor{rulegray}{gray}{0.7}			% Listings rule/frame color

% Style for Verilog
\lstdefinestyle{Verilog}{
	language=Verilog,					% Verilog
	backgroundcolor=\color{listinggray},	% light gray background
	rulecolor=\color{blue}, 			% blue frame lines
	frame=tb,							% lines above & below
	linewidth=\columnwidth, 			% set line width
	basicstyle=\small\ttfamily,	% basic font style that is used for the code	
	breaklines=true, 					% allow breaking across columns/pages
	tabsize=3,							% set tab size
	commentstyle=\color{gray},	% comments in italic 
	stringstyle=\upshape,				% strings are printed in normal font
	showspaces=false,					% don't underscore spaces
}

% How to use: \Verilog[listing_options]{file}
\newcommand{\Verilog}[2][]{%
	\lstinputlisting[style=Verilog,#1]{#2}
}




%======================================================
%=========== Body  ====================================
\begin{document}

\title{ELC 2137 Lab \#3: Adders}
\author{Sebastian Lopez and Megan Gordon}

\maketitle


\section*{Summary}

In this lab we developed circuits that implement a given logic function, described the operation of half, full, and ripple adders, and we developed a moderately complex circuit on a breadboard using standard electrical parts. We also tested and verified the operation of each circuit. 

\section*{Q\&A}

\begin{table}[ht]\centering
	\caption{Half Adder Truth Table}
	\label{tbl:Logic_Truth_Table}
	\begin{tabular}{cc||cc||c}
		\toprule
		A & B & C & S & Decimal\\
		\midrule
		0 & 0 & 0 & 0 & 0\\
		0 & 1 & 0 & 1 & 1\\
		1 & 0 & 0 & 1 & 1\\
		1 & 1 & 1 & 0 & 2\\
		\bottomrule
	\end{tabular} 
\end{table} 

\begin{enumerate}
	\item Which gates could we use for combining the carry bits? 
	      Either an XOR or AND gate. 
	\item Which one should we use and why? 
		  Although we can use either XOR or AND, XOR is an overall better gate to use considering the fact that it is more efficient. 
\end{enumerate}

\section*{Results}

\begin{figure}[ht]\centering
	\includegraphics[width=0.75\textwidth, angle = 270]{"Circuit Demonstration Page ".jpg}
	\caption{This is the Circuit Demonstration Page.}
	\label{fig:circuit_demonstration}			% label must be after caption
\end{figure}

\begin{figure}[ht]\centering
	\includegraphics[width=0.75\textwidth, angle = 270]{"Half Adder".jpg}
	\caption{This is the Half Adder.}
	\label{fig:half_adder}			% label must be after caption
\end{figure}

\begin{figure}[ht]\centering
	\includegraphics[width=0.75\textwidth, angle = 270]{"Full Adder".jpg}
	\caption{This is the Full Adder.}
	\label{fig:full_adder}			% label must be after caption
\end{figure}

\begin{figure}[ht]\centering
	\includegraphics[width=0.7\textwidth]{"2-bit Adder".jpg}
	\caption{This is the 2-bit Adder.}
	\label{fig:2-bit_adder}			% label must be after caption
\end{figure}



\end{document}
