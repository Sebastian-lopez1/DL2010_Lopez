% Digital Logic Report Template
% Created: 2020-01-16 Sebastian Lopez 

%==========================================================
%=========== Document Setup  ==============================

% Formatting defined by class file
\documentclass[11pt]{article}

% ---- Document formatting ----
\usepackage[margin=1in]{geometry}	% Narrower margins
\usepackage{booktabs}				% Nice formatting of tables
\usepackage{graphicx}				% Ability to include graphics

%\setlength\parindent{0pt}	% Do not indent first line of paragraphs 
\usepackage[parfill]{parskip}		% Line space b/w paragraphs
%	parfill option prevents last line of pgrph from being fully justified

% Parskip package adds too much space around titles, fix with this
\RequirePackage{titlesec}
\titlespacing\section{0pt}{8pt plus 4pt minus 2pt}{3pt plus 2pt minus 2pt}
\titlespacing\subsection{0pt}{4pt plus 4pt minus 2pt}{-2pt plus 2pt minus 2pt}
\titlespacing\subsubsection{0pt}{2pt plus 4pt minus 2pt}{-6pt plus 2pt minus 2pt}

% ---- Hyperlinks ----
\usepackage[colorlinks=true,urlcolor=blue]{hyperref}	% For URL's. Automatically links internal references.

% ---- Code listings ----
\usepackage{listings} 					% Nice code layout and inclusion
\usepackage[usenames,dvipsnames]{xcolor}	% Colors (needs to be defined before using colors)

% Define custom colors for listings
\definecolor{listinggray}{gray}{0.98}		% Listings background color
\definecolor{rulegray}{gray}{0.7}			% Listings rule/frame color

% Style for Verilog
\lstdefinestyle{Verilog}{
	language=Verilog,					% Verilog
	backgroundcolor=\color{listinggray},	% light gray background
	rulecolor=\color{blue}, 			% blue frame lines
	frame=tb,							% lines above & below
	linewidth=\columnwidth, 			% set line width
	basicstyle=\small\ttfamily,	% basic font style that is used for the code	
	breaklines=true, 					% allow breaking across columns/pages
	tabsize=3,							% set tab size
	commentstyle=\color{gray},	% comments in italic 
	stringstyle=\upshape,				% strings are printed in normal font
	showspaces=false,					% don't underscore spaces
}

% How to use: \Verilog[listing_options]{file}
\newcommand{\Verilog}[2][]{%
	\lstinputlisting[style=Verilog,#1]{#2}
}




%======================================================
%=========== Body  ====================================
\begin{document}

\title{ELC 2137 Lab \#4: Subtractor}
\author{Sebastian Lopez and Megan Gordon}

\maketitle


\section*{Summary}

In this lab we described the operation of a two-bit adder/subtractor. We developed a moderately complex circuit on a breadboard using standard electrical parts, as well as our own test procedure and verification operation of the circuit. Then, we recognized that the digital circuits quickly become complex and difficult to implement in hardware. 

\section*{Q\&A}

\begin{enumerate}
	\item Why did we use two full adders instead of a half adder and a full adder? 
	
	\item How many input combinations would it take to exhaustively test the adder/subtractor? 
	It would take six input combinations to exhaustively test the adder/subtractor.
	\item Why were the combinations given in the truth table chosen?
	The reason why these specific combinations were given is due to the fact that it gives the proper values to effectively run and test the adder/subtractor. 
	\item Do the results from your adder/subtractor match what you would expect from theory? Explain any discrepancies. 
	The c values when calculated were incorrect in comparison to that of the expected results. The c values needed to be flipped in order to create the correct results for the adder/subtraactor. 
	
\end{enumerate}

\section*{Results}

\begin{figure}[ht]\centering
	\includegraphics[width=1\textwidth]{"Assembled_Circuit".jpg}
	\caption{This is the assembled circuit.}
	\label{fig:circuit_demonstration}			% label must be after caption
\end{figure}

\begin{figure}[ht]\centering
	\includegraphics[width=1\textwidth, angle = 270]{"Two-bit_adder_subtractor_schematic".jpg}
	\caption{This is the two bit adder/subtractor schematic.}
	\label{fig:half_adder}			% label must be after caption
\end{figure}

\begin{figure}[ht]\centering
	\includegraphics[width=1\textwidth, angle = 270]{"Circuit_Demonstration_Page".jpg}
	\caption{This is the circuit demonstration page.}
	\label{fig:full_adder}			% label must be after caption
\end{figure}

\end{document}
